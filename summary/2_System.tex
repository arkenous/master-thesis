% @suppress JapaneseAmbiguousNounConjunction CommaNumber JapaneseNumberExpression
\section{システム概要}
本システムは,Android端末に一般的に搭載されている加速度センサと角速度センサを用いてモーションデータ(以下,データ)を収集し,人工ニューラルネットワークの一つであるDenoising Autoencoder(以下,dA)とその後ろに識別用ニューロンを繋げた識別器を用いて個人認証を行う.
%データの取得間隔はAndroid APIで用意された``SENSOR\_DELAY\_FASTEST''を指定しており,本システムの開発時に使用したNexus5ではおよそ5ミリ秒間隔でデータが取得できていることを確認した.
本システムではモーション入力を任意の時間で行えるが,これによるデータ数の差異を解決するため登録モードでは入力時間の最も長かったものを,認証モードでは登録時に用いたものを基準にゼロによるパディングか末尾の切り落としを行う.
また,フーリエ変換を用いたローパスフィルタによりデータ入力時に生じる手の震えによる影響を減少させる.
さらに,加速度から変位,角速度から角度に変換したのち変位データを角度データで回転させたものを用いて以降の処理を行う.

登録モードでは,Android端末で動作するアプリケーション(以下,クライアント)から,モーションの入力を任意回数で行える.
モーションが入力され前述のデータ加工をしたのち,CUDAサーバで動作するプログラム(以下,サーバ)にデータを送信する.
サーバでは,受信したデータについて平均が0,分散が1になるように正規化する.
そして,正規化したデータの30\%に平均が0で分散が1のガウシアンノイズによるノイズ加工を行う.
ノイズ加工を行った後,中間層のニューロン数を入力層や出力層のニューロン数から50\%削減し,中間層の活性化関数にシグモイド関数,出力層の活性化関数に恒等関数を用いたdAを構築する.
訓練データにノイズ加工を行ったデータを,教師信号にノイズ加工を行う前のデータを与え,損失関数に最小二乗誤差を用いて300回の学習を行う.
また,訓練時に中間層ニューロンのうち50\%をランダムにDropoutさせる.
dAの学習が終わればパラメータを固定して,その後ろに活性化関数にシグモイド関数を用いたニューロンを繋げる.
そして,正規化する前のデータの値域で生成した乱数でデータの20\%を上書きしたダミーデータを生成し,正規化する.
訓練データに正規化したデータとダミーデータを,教師信号にそれぞれ0.0と1.0を与え,損失関数に交差エントロピー誤差を用いて誤差が0.1未満になるまで最大1000回の学習を行う.
%この際Dropoutを無効にし,dAの出力に対して1から先ほどの学習時に適用したDropout率を引いた0.5を掛ける.
学習が終われば識別器を構成するニューロンが持つパラメータを文字列として連結したデータをクライアントに送信する.
クライアント側は受信したパラメータを暗号化し,他アプリからの読み書きができない形で保存する.

認証モードでは,クライアントからのモーション入力は1回のみ行える.
モーションが入力され前述のデータ加工をしたのち,サーバにデータと保存したパラメータを送信する.
サーバでは,受信したデータを平均が0,分散が1になるように正規化する.
次に,受け取った学習済みニューラルネットワークのパラメータを元に識別器を構築する.
構築できたらこれに正規化したデータを与え,得られた出力値をクライアントに送信する.
クライアント側はこの値を受け取り,値が0.3未満であれば認証成功とし,0.3以上であれば認証失敗とする.
また,認証モードに限り,クライアントが何らかの理由でサーバに接続できない場合はクライアント側でサーバと同様の処理が行える.
