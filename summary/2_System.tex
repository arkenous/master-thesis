\section{システム概要}
本研究で開発したシステム(以下,本システム)は,Android端末に一般的に搭載されている加速度センサと角速度センサを用いてモーションデータを収集し,人工ニューラルネットワークの一つであるDenoising Autoencoderとその後ろに識別用ニューロンを繋げた識別器を用いて個人認証を行う.
モーションデータの入力は任意の時間で行うことができる.
モーションデータの取得間隔はAndroid APIで用意された``SENSOR_DELAY_FASTEST''を指定しており,本システムの開発時に使用したNexus5ではおよそ5ミリ秒間隔でデータが取得できていることを確認した.

登録モードでは,Android端末で動作するアプリケーション(以下,クライアント)から,モーションの入力を任意回数で行える.
モーションが入力されたら,後述するモーションデータの加工をしたのちGPGPUの一つであるCUDAを用いた高速演算が可能な計算機上で動作するサーバプログラム(以下,サーバ)にTCPソケットを用いてモーションデータを送信する.
サーバでは,受信したモーションデータのそれぞれに対して平均が0,分散が1になるように正規化する.
そして,正規化したモーションデータのそれぞれ40\%にそのデータの最大値あるいは最小値で上書きするノイズ加工を行う.
ノイズ加工を行った後,中間層のニューロン数を入力層や出力層のニューロン数から30\%削減し,中間層の活性化関数にシグモイド関数,出力層の活性化関数に恒等関数を用いたDenoising Autoencoderを構築する.
訓練データにノイズ加工を行ったモーションデータを,教師信号にノイズ加工を行う前のモーションデータを与え,損失関数に最小二乗誤差を用いて500回の学習を行う.
また,訓練時に中間層ニューロンのうち50\%をランダムにDropoutさせる.

Denoising Autoencoderの学習が終われば,Denoising Autoencoderのパラメータを固定してその後ろに活性化関数にシグモイド関数を用いたニューロンを繋げる.
そして,正規化する前のモーションデータにおける最大値と最小値の範囲内で生成したランダム値からなるダミーデータを生成し,正規化する.
訓練データに正規化したモーションデータとダミーデータを,教師信号にそれぞれ0.0と1.0を与え,損失関数に交差エントロピー誤差を用いて誤差が0.1未満になるまで最大10000回の学習を行う.
この際Dropoutを無効にし,Denoising Autoencoderの出力に対して1から先ほどの学習時に適用したDropout率を引いた0.5を掛ける.

学習が終われば識別器を構成するニューロンがそれぞれ持つパラメータを文字列として連結したデータをクライアントに送信する.
クライアント側は受信したパラメータを暗号化し,他アプリからの読み書きができない形で保存する.

認証モードでは,クライアントからのモーション入力は1回のみ行える.
モーションが入力されたら,後述するモーションデータの加工をしたのちサーバにモーションデータと保存したパラメータをTCPソケットを用いて送信する.
サーバでは,受信したモーションデータを平均が0,分散が1になるように正規化する.
次に,受け取った学習済みニューラルネットワークのパラメータをもとに識別器を構築する.
構築できたらこれに正規化したモーションデータを与え,得られた出力値をクライアントに送信する.

クライアント側はこの値を受け取り,値が0.2未満であれば認証成功とし,0.2以上であれば認証失敗とする.

また,認証モードに限り,クライアントが何らかの理由でサーバに接続できない場合はクライアントのみでサーバと同様の処理が行えるようにしている.

% @suppress
\subsection{モーションデータの加工}

