\chapter{序論}

\section{研究背景}
% スマホ端末の急速な普及,それに伴う個人情報漏洩等のリスク
近年,スマートフォンと呼ばれる携帯デバイスが急速に普及しつつある.
スマートフォンとは,パーソナルコンピュータ向けに設計されたWebサイトを閲覧できる機能を持つフルブラウザを搭載し,様々な企業や個人が開発した多種多様なアプリケーションをインストールし利用できる携帯デバイスのことを指す\cite{1-smartphone}.
平成28年版の情報通信白書によるとスマートフォンの世帯普及率は2015年末時点で72.0\%とあり,また前年比で7.8ポイント増となっている\cite{1-spread}.
スマートフォンの普及によりどこでも手軽にオンラインショッピングやネットバンキングをはじめとする多種多様なサービスを利用できるようになった.

その一方で,これらサービスの利用にはユーザIDやパスワード等を含む個人情報を用いた個人認証を必要とする場合が多い.
また,利用しているブラウザやアプリケーションによっては,サービスにログインすれば一定期間ログイン状態を保持し再ログインの手間を省くような機能を持つものもある.
この機能により,ユーザはサービスを利用するたびに再ログインする手間が無くなることから利便性が向上する.
しかしながら,悪意のある第三者がサービスへのログインに必要な情報を知らずとも,本来のユーザになりすましてサービスを利用できてしまうという危険性がある.

このように,スマートフォンは従来型のフィーチャーフォンと比較してより多くの個人情報を内包しており,第三者からのこれら情報への不正なアクセスを防ぐための仕組みが不可欠となっている.
現在この仕組みを実現する方法として広く採用されているのが,端末利用時にあらかじめ登録したパスコード情報や指紋情報をもとに,現在の利用者が本来の端末所有者であるかを確認する個人認証システムである.



% スマホ内の個人情報等を保護するための個人認証システムとその課題点

\section{研究目的}
% 前述した課題に対する本研究のアプローチ

\section{本論文の構成}
% 第2章以降の簡単な説明
