% @suppress VoidSection
\chapter{序論}

% @suppress HankakuKana KatakanaSpellCheck
\section{研究背景}
% スマホ端末の急速な普及,それに伴う個人情報漏洩等のリスク
近年,スマートフォンと呼ばれる携帯端末が急速に普及しつつある.
スマートフォンとは,パーソナルコンピュータ向けに設計されたWebサイトを閲覧できる機能を持つフルブラウザを搭載し,様々な企業や個人が開発した多種多様なアプリケーションをインストールし利用できる携帯端末のことを指す\cite{1-smartphone}.
平成28年版の情報通信白書によるとスマートフォンの世帯普及率は2015年末時点で72.0\%とあり,また前年比で7.8ポイント増となっている\cite{1-spread}.
スマートフォンの普及によりどこでも手軽にオンラインショッピングやネットバンキングをはじめとする多種多様なサービスを利用できるようになった.

その一方で,これらサービスの利用にはユーザIDやパスワード等を含む個人情報を用いた個人認証を必要とする場合が多い.
また,利用しているブラウザやアプリケーションによっては,サービスにログインすれば一定期間ログイン状態を保持し再ログインの手間を省くような機能を持つものもある.
この機能により,ユーザはサービスを利用するたびに再ログインする手間が無くなることから利便性が向上する.
しかしながら,悪意のある第三者がサービスへのログインに必要な情報を知らずとも,本来のユーザになりすましてサービスを利用できてしまうという危険性がある.

% スマホ内の個人情報等を保護するための個人認証システムとその課題点
このように,スマートフォンは従来型のフィーチャーフォンと比較してより多くの個人情報を内包しており,第三者からのこれら情報への不正なアクセスを防ぐための仕組みが不可欠となっている.
現在この仕組みを実現する方法として広く採用されているのが,端末利用時にあらかじめ登録したパスコード情報や指紋情報をもとに,現在の利用者が本来の端末所有者であるかを確認する個人認証システムである.
パスコード認証方式では,あらかじめ端末所有者が特定の文字種からパスコードを構築し,これを端末に登録しておく.
そして,端末利用時に入力されたパスコードと登録されたパスコードを比較して同一であれば端末所有者であるとみなして,その後の端末利用を許可する.
指紋認証方式では,あらかじめ端末所有者が端末に搭載された指紋スキャナを通じて自らの指紋をスキャンし,これを端末に登録しておく.
そして,端末利用時に指紋をスキャンして登録された指紋との比較をし,同一であれば端末所有者であるとみなしてその後の端末利用を許可する.
これらの個人認証システムを利用することにより,第三者によって不正に端末内の個人情報へアクセスされる危険性をある程度軽減できる.
しかし,これらの認証方式にはそれぞれいくつかの問題点が挙げられる.

まずパスコード認証方式だが,これは個人認証を行う際にスマートフォン画面上に表示されたソフトウェアキーボードを目視し指でタッチして操作する必要があり,ユーザにとっては煩雑である可能性があるという点がある.
またあらかじめ決められた文字種の中から一つずつ選んだ文字を並べてパスコードを構築することから,パスコードのパターン数が限られ,認証に用いる鍵の自由度が制限されてしまうという点がある.

指紋認証方式については指紋をスキャンするためのハードウェアをスマートフォンに搭載しなければならないという点がある.
また指紋情報は変更ができないため,何らかの原因でこの情報が第三者に漏洩した場合は,今後その指紋を用いた個人認証ができなくなるという点がある.
さらに,ドイツのハッカー集団であるChaos Computer Clubの生体認証チームが,一般的なカメラで撮影された写真に写り込んだ指から指紋を複製することに成功している\cite{1-ccc}.
このことから,指紋情報が漏洩する可能性が十分にあり個人認証システムが担う機密性の確保が難しいといえる\cite{1-sophos}.

\section{研究目的}
% 前述した課題に対する本研究のアプローチ
本研究では,一般的なスマートフォンに搭載されている加速度センサと角速度センサを利用し,端末を振る動き(以下,モーション)で個人認証を行うシステムを開発する.
これによりパスコード認証方式における認証作業の煩雑さと鍵情報の自由度が制限されるという課題点を軽減し,指紋認証方式における指紋情報が漏洩した場合に鍵情報の変更ができないという課題点を解消した生体認証システムの実現を目指す.

\section{本論文の構成}
% 第2章以降の簡単な説明
第2章にて本研究に関連する先行研究について述べる.
第3章では本研究で開発した個人認証システムを提案するにあたり必要となる知識について説明する.
第4章では本研究で開発した個人認証システムの実装について,その詳細を述べる.
第5章では本研究で開発した個人認証システムの評価実験とその結果を示し,第6章で結論と今後の課題を述べたあと,本論文を総括する.
