\chapter{提案システム}
本章では,本研究で開発したスマートフォンのモーションセンサを利用した個人認証システムについて説明する.

\section{システムの概要}
本システムは,Android端末に一般的に搭載されている加速度センサと角速度センサを用いてモーションデータを収集し,Denoising Autoencoderとその後ろに識別用ニューロンを繋げた識別器を用いて個人認証を行う.
本システムには,登録モードと認証モードの二つの動作モードがある.
登録モードでは,入力されたモーションデータの一部をランダムにデータの最大値及び最小値でノイズを付与したものを用いてDenoising Autoencoderで特徴を学習させる.
その後識別用ニューロンを繋いで入力データに対する教師信号として0.0を,入力データと同じ値域・次元数で乱数生成したダミーデータに対する教師信号として1.0を与えて識別器の学習を行う.
認証モードでは,入力されたモーションデータを学習済みの識別器に入力し,得られた出力を用いて個人認証を行う.

\subsection{動作モード選択}
Androidアプリケーションを起動すると,図\ref{start}のスタート画面が表示される.
画面下部にある``Start''ボタンを押すことで,図\ref{select-mode}の動作モード選択ダイアログが表示される.
``Registration''を選択すると登録モードに,``Authentication''を選択すると認証モードに遷移する.

\subsection{登録モード}
登録モードに遷移すると,図\ref{reg-input-name}のユーザ名入力画面が表示される.
ここでは登録するユーザ名を入力する.
ユーザ名を入力して画面下部にある``OK''ボタンを押すことでユーザ名が正しく入力されたか確認する処理が行われ,確認できれば図\ref{registration}のモーション入力画面が表示される.
ユーザ名が入力されていない,もしくは空白文字しか入力されていない場合は図\ref{reg-input-name-toast}のようにAndroidシステムに標準で用意されている通知機能であるToastを用いてユーザにエラーを通知する.

モーション入力画面では,画面下部の``モーションデータ取得''ボタンを押している間,加速度センサと角速度センサそれぞれにデータ取得のための専用スレッドが起動して各センサからデータが取得・蓄積される.
また,データ取得用スレッドの起動と同時に時間計測用スレッドも起動し,1秒経過毎に端末をバイブレートさせることでユーザにモーション入力の経過時間を伝える.
モーション入力中に任意のタイミングで``モーションデータ取得''ボタンから指を離すことで,モーション入力を終了できる.
モーション入力はデフォルトでは3回となっているが,画面右上のハンバーガーメニュー,もしくは端末に搭載されたメニューボタンを押すことで表示される図\ref{reg-menu}のメニュー画面から,``データ取得回数設定''を選択することで回数を変更できる.
``データ取得回数設定''を選択すると,図\ref{reg-change-time}のダイアログが表示される.
画面上にあるスライダーを操作して左右に動かすことで,データ取得回数を増減でき,画面下部の``OK''ボタンを押すことで設定が反映される.
また,先ほどのメニュー画面から``リセット''を選択すると,図\ref{reg-reset}のダイアログが表示される.
ここで``YES''を選択することで,モーションデータの取得状態をリセットし,一からモーション入力をやり直すことができる.

モーション入力が終わるたびに,取得したモーションデータのデータ数が確認される.
加速度センサ及び角速度センサから得られたX軸データのいずれかのデータ数が10個を下回っていた場合,図\ref{reg-recollect}のダイアログを表示し,ユーザに再度モーションを入力させる.
設定回数分のモーション入力が終わると図\ref{reg-progress}のプログレスダイアログが表示され,後述するモーションデータの加工及び識別器の学習を行うスレッドが起動する.

モーションデータの加工及び識別器の学習が終わると図\ref{reg-finish}のダイアログが表示される.
``OK''ボタンを押すことでユーザ名と暗号化した学習済み識別器のパラメータ,データの次元数を他アプリからの読み書きができない形で端末に保存し,スタート画面に遷移する.
何らかの原因で識別器の学習ができなかった場合は図\ref{reg-error}のダイアログが表示される.
``OK''ボタンを押すことでモーションデータの取得状態がリセットされるので再度モーションを入力する.

\subsection{認証モード}
認証モードに遷移すると,図\ref{auth-input-name}のユーザ名入力画面が表示される.
ここでは認証するユーザ名を入力する.
ユーザ名を入力して画面下部にある``OK''ボタンを押すことで登録モードで保存されたユーザ名のリストから入力されたユーザ名と合致するものが存在するか検索され,存在していた場合は図\ref{authentication}のモーション入力画面が表示される.
存在していなかった場合は,図\ref{auth-input-name-toast}のようにToastを用いてユーザにエラーを通知する.

モーション入力画面では,画面下部の``モーションデータ取得''ボタンを押している間,加速度センサと角速度センサそれぞれにデータ取得のための専用スレッドが起動して各センサからデータが取得・蓄積される.
また,データ取得用スレッドの起動と同時に時間計測用スレッドも起動し,1秒経過毎に端末をバイブレートさせることでユーザにモーション入力の経過時間を伝える.
モーション入力中に任意のタイミングで``モーションデータ取得''ボタンから指を離すことで,モーション入力を終了できる.
モーション入力は1回となっており,登録モードと違い変更はできない.

モーション入力が終わると,取得したモーションデータのデータ数が確認される.
加速度センサ及び角速度センサから得られたX軸データのいずれかのデータ数が10個を下回っていた場合,図\ref{auth-recollect}のダイアログを表示し,ユーザに再度モーションを入力させる.
設定回数分のモーション入力が終わると図\ref{auth-progress}のプログレスダイアログが表示され,後述するモーションデータの加工及び識別器による個人認証を行うスレッドが起動する.

モーションデータの加工及び個人認証が終わると,認証結果をダイアログで表示する.
認証に成功すれば図\ref{auth-succeed}のダイアログが表示され,``OK''ボタンを押すことでスタート画面に遷移する.
認証に失敗すれば図\ref{auth-failure}のダイアログが表示され,``OK''ボタンを押すことでモーションデータの取得状態がリセットされ,再度個人認証を行える.

\section{モーションデータの加工}
登録モードと認証モードのいずれも,モーションセンサから得られたデータはニューラルネットワークで用いる前に,``フーリエ変換を用いたローパスフィルタ'',``角速度から変位,角速度から角度への変換'',``変位データを角度データで回転''という三つの加工を行う.

\subsection{ローパスフィルタ}
モーションを入力している際の手の震えなどによるモーションデータへの影響を抑えるために,フーリエ変換を用いたローパスフィルタ処理を実装している.
時間軸領域で表されるデータをフーリエ変換を用いて周波数領域に変換すると,モーション入力中の手の震えなどによるデータが高周波成分として現れる.
この高周波成分を取り除いた上で元の時間軸領域のデータに逆変換するローパスフィルタ処理を行うことで,手の震えなどによる影響の少ないデータを得られる.

フーリエ変換の実装には,CERNのColt Project\cite{4-colt-project}で開発された科学技術計算用ライブラリであるColtをマルチスレッド化したParallel Colt\cite{4-parallel-colt}に含まれている,JTransforms\cite{4-jtransforms}を用いた.

この処理をソースコード\ref{source-lowpass}に示す.

\lstinputlisting[caption=ローパスフィルタ, label=source-lowpass, style=MyJava]{Code/lowpass.java}

% ローパスフィルタの比較グラフ
ローパスフィルタ処理によるデータの変化を示したグラフを図\ref{graph-lowpass}に示す.

\begin{figure}[hbtp]
  \centering
  \includegraphics[bb=0 0 360 216, width=12cm]{Graphs/lowpass.pdf}
  \caption{ローパスフィルタ処理によるデータの変化}
  \label{graph-lowpass}
\end{figure}

青色で示した線がローパスフィルタ処理前のグラフ,赤色で示した線が処理後のグラフである.

\subsection{加速度から変位,角速度から角度への変換}

\subsection{変位データを角度データで回転}


% @suppress
\section{ニューラルネットワークによる学習と識別}
