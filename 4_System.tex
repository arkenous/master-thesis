\chapter{提案システム}
本章では,本研究で開発したスマートフォンのモーションセンサを利用した個人認証システムについて説明する.

\section{システムの概要}
本システムは,Android端末に一般的に搭載されている加速度センサと角速度センサを用いてモーションデータを収集し,Denoising Autoencoderとその後ろに識別用ニューロンを繋げた識別器を用いて個人認証を行う.
本システムには,登録モードと認証モードの二つの動作モードがある.
登録モードでは,入力されたモーションデータの一部をランダムにデータの最大値および最小値でノイズを付与したものを用いてDenoising Autoencoderで特徴を学習させる.
その後識別用ニューロンを繋いで入力データに対する教師信号として0.0を,入力データと同じ値域・次元数で乱数生成したダミーデータに対する教師信号として1.0を与えて識別器の学習を行う.
認証モードでは,入力されたモーションデータを学習済みの識別器に入力し,得られた出力を用いて個人認証を行う.

\subsection{動作モード選択}
Androidアプリケーションを起動すると,図\ref{start}の画面が表示される.
画面下部にある``Start''ボタンを押すことで,図\ref{select-mode}の動作モード選択ダイアログが表示される.
``Registration''を選択すると登録モードに,``Authentication''を選択すると認証モードに遷移する.

\subsection{登録モード}
登録モードに遷移すると,図\ref{reg-input-name}の画面が表示される.
ここでは登録するユーザ名を入力する.
ユーザ名を入力して画面下部にある``OK''ボタンを押すことで次のモーション入力画面に遷移するが,ユーザ名が入力されていない,もしくは空白文字しか入力されていない場合は図\ref{reg-input-name-toast}のようにAndroidシステムに標準で用意されている通知機能であるToastを用いてユーザに再入力を促す.

ユーザ名が正しく入力された状態で``OK''ボタンを押すと,図\ref{registration}の画面が表示される.
ここではモーションの入力を行う.
画面下部の``モーションデータ取得''ボタンを押している間,加速度センサと角速度センサそれぞれにデータ取得のための専用スレッドが起動し,各センサからデータが取得・蓄積される.
また,データ取得用スレッドの起動と同時に時間計測用スレッドも起動し,1秒経過毎に端末をバイブレートさせることでユーザにモーション入力の経過時間を伝える.
モーション入力中に任意のタイミングで``モーションデータ取得''ボタンから指を離すことで,モーション入力を終了できる.
モーション入力はデフォルトでは3回となっているが,画面右上のハンバーガーメニュー,もしくは端末に搭載されたメニューボタンを押すことで表示される図\ref{reg-menu}のようなメニュー画面から,``データ取得回数設定''を選択することで回数を変更できる.
``データ取得回数設定''を選択すると,図\ref{reg-change-time}のような画面が表示される.
画面上にあるスライダーを操作して左右に動かすことで,データ取得回数を増減でき,画面下部の``OK''ボタンを押すことで設定が反映される.
また,先ほどのメニュー画面から``リセット''を選択すると,図\ref{reg-reset}のような画面が表示される.
ここで``YES''を選択することで,モーションデータの取得状態をリセットし,一からモーション入力をやり直すことができる.

モーション入力が終わるたびに,取得したモーションデータのデータ数が確認される.
加速度センサ及び角速度センサから得られたX軸データのいずれかのデータ数が10を下回っていた場合,図\ref{reg-recollect}のようなダイアログを表示し,ユーザに再度モーションを入力させる.
設定回数分のモーション入力が終わると図\ref{reg-progress}のようなプログレスダイアログが表示され,データ処理および識別器学習用スレッドが起動する.



% @suppress
\subsection{認証モード}

% @suppress
\section{モーションデータの取得}

% @suppress
\section{モーションデータの加工}

% @suppress
\section{人工ニューラルネットワークによる学習と識別}
