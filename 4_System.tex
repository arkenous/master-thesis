\chapter{提案システム}
本章では,本研究で開発したスマートフォンのモーションセンサを利用した個人認証システムについて説明する.

\section{システムの概要}
本システムは,Android端末に一般的に搭載されている加速度センサと角速度センサを用いてモーションデータを収集し,Denoising Autoencoderとその後ろに識別用ニューロンを繋げた識別器を用いて個人認証を行う.
本システムには,登録モードと認証モードの二つの動作モードがある.
登録モードでは,入力されたモーションデータの一部をランダムにデータの最大値および最小値でノイズを付与したものを用いてDenoising Autoencoderで特徴を学習させる.
その後識別用ニューロンを繋いで入力データに対する教師信号として0.0を,入力データと同じ値域・次元数で乱数生成したダミーデータに対する教師信号として1.0を与えて識別器の学習を行う.
認証モードでは,入力されたモーションデータを学習済みの識別器に入力し,得られた出力を用いて個人認証を行う.


% @suppress
\section{システムの構成}

% @suppress
\subsection{登録モード}

% @suppress
\subsection{認証モード}

% @suppress
\section{モーションデータの取得}

% @suppress
\section{モーションデータの加工}

% @suppress
\section{人工ニューラルネットワークによる学習と識別}
