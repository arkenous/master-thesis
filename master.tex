%
% $Id: blank-thesis.tex,v 1.1 2014-05-21 16:07:38+09 kobayasi Exp $
%
\documentclass[12pt]{jreport}
\usepackage{newcent}             % PDFへの変換後の品質を高める %なくても大丈夫らしい.
\usepackage[dvipdfmx]{graphicx}
\usepackage{comment}
\usepackage{url}
\usepackage{fancyhdr}
\usepackage{color}
\definecolor{purple}{rgb}{0.6,0,0.4}
\definecolor{brown}{cmyk}{0,0.81,1,0.60}
\definecolor{gray}{rgb}{0.4,0.4,0.4}
\definecolor{darkblue}{rgb}{0.0,0.0,0.6}
\definecolor{cyan}{rgb}{0.0,0.6,0.6}
\usepackage{listings, jlisting}
\lstdefinestyle{MyJava} % JavaとXML,2つの言語を使いたかったので.style=MyJavaとかで使い分け
{
language=Java, % lstlisting内の言語の指定
numbers=left, % 行番号を左端に表示する
breaklines = true, % 行が長くなってしまった場合の改行を行う
basicstyle={\small}, % 標準の書式設定
identifierstyle={\small}, % 識別子のスタイル
keywordstyle={\small\bfseries\color{purple}}, % キーワードのスタイル
commentstyle={\small\itshape\color{gray}}, % コメントのスタイル
stringstyle={\small\ttfamily\color{brown}}, % 文字列のスタイル
frame=single, % 枠のスタイル
tabsize=2, % タブ幅
showstringspaces=false, % 空白を可視化するか
}
\lstdefinelanguage{XML} % XMLはあまり充実してないってさー
{
    morestring=[b]",
    morestring=[s]{>}{<},
    morecomment=[s]{<?}{?>},
    stringstyle=\color{brown},
    identifierstyle=\color{darkblue},
    keywordstyle=\color{cyan},
    morekeywords={xmlns,version,type}% list your attributes here
}
\lstdefinestyle{MyXML} % JavaとXM(ry
{
language=XML,
numbers=left,
breaklines=true,
basicstyle={\small},
identifierstyle={\small\color{darkblue}},
keywordstyle={\small\bfseries\color{cyan}},
commentstyle={\small\itshape\color{gray}},
stringstyle={\small\ttfamily\color{brown}},
frame=single,
tabsize=2,
showstringspaces=false,
}

\renewcommand{\slash}{/}
\graphicspath{{./Screenshots/}, {./Graphs/}, {./Figures/}} % 図用の画像ファイルの保存パスを指定
\usepackage{otf} % OpenType Font utfなどにしか無いような文字を利用する
\usepackage{longtable} % ページをまたぐ表の作成
\usepackage[master]{grad}

\title{\bfseries スマートフォンのモーションセンサを利用した\\個人認証アプリケーションの開発}
\graduate{総合情報学研究科\\知識情報学専攻}
\department{マルチメディア情報システムの基礎と実際}
\id{15M7112}
\author{\UTF{9AD9}坂 賢佑}
\date{}%空のままにすること

\renewcommand{\bibname}{参考文献,参考URL等}

\begin{document}
\maketitle

\pagenumbering{roman}
\chapter*{要旨}
スマートフォンにおける既存の認証方式として,パスコード認証や指紋認証がある.これら認証方式はユーザにとって比較的馴染みの深いものであるが,一方で様々な欠点が存在する.本論文ではこれら認証方式に代わり,スマートフォンに一般的に搭載されている加速度センサと角速度センサを用いて端末を振ることで個人を認証するアプリケーションを開発し,これを提案する.



\tableofcontents %目次作成(必須)
\listoffigures   %図目録(任意)
\listoftables    %表目録(任意)

\clearpage
\setcounter{page}{0}
\pagenumbering{arabic}

%\section{はじめに}
近年,スマートフォンの普及により日々の生活がより豊かになった一方で,端末がより多くの個人情報を内包するようになった.
このことから,第三者によって不正に個人情報にアクセスされたり,インストールされたアプリを通じて様々なサービスをなりすまし利用されたりといった場合の危険性はより高くなった.
そのため,端末利用時には本来の端末所有者であるか認証するよう設定することが推奨されている.
現在スマートフォンにおいて個人の認証方式として広く使われている方法としてパスコード認証と指紋認証がある.
これら認証方式には,パスコード認証には認証作業が煩雑で認証に用いる鍵情報の自由度が低いということが,指紋認証には鍵情報が変更できないため何らかの原因で鍵情報が第三者に漏洩した場合に今後その情報を用いた個人認証ができなくなるということが問題点として挙げられる.

本研究ではスマートフォンに一般的に搭載されている加速度センサと角速度センサを利用し,人間の動き(以下,モーション)を用いて個人認証を行うシステムを開発することで,既存の認証方式が抱える問題点の解決を目指す.

%\input{7_Conclusion.tex}
%\clearpage
%\chapter*{謝辞}
本研究における個人認証システムの開発及び本稿の執筆にあたり,関西大学総合情報学部の小林孝史准教授,桑門秀典教授,堀井康史教授,田頭茂明教授に深く御礼申し上げます.
特に,小林孝史准教授には学部生の頃から毎週のゼミを中心に日頃から多くの御指導を賜り,御多忙の中時間を割いて下さったことに対し,心より感謝いたします.
また,小林研究室に所属する学部生と大学院生,先輩方,また総合情報学研究科で共に学んだ同輩の方々にも様々なご協力を頂きました.
特に,小林研究室に入り,研究テーマに悩んでいた際に先輩方からアドバイスを頂いたからこそ,この研究一筋で没頭することができました.
この場を借りて,御礼申し上げます.

また,大学院の講義でニューラルネットワークを教えて下さった林勲教授に対して,御礼を申し上げます.
講義を受けるまでニューラルネットワークの仕組みについてほぼ何も知らなかった私に対し,1から噛み砕いて,手を動かしつつ分かりやすく教えて下さったからこそ,本研究でもニューラルネットワークをシステムに組み込むことができました.

%
\renewcommand{\headrulewidth}{0pt}
\pagestyle{fancy}
\lhead{参考文献,参考URL等}
\chead{\empty}
\rhead{\thepage}
\lfoot{\empty}
\cfoot{\empty}
\rfoot{\empty}
\begin{thebibliography}{30}
\bibitem{1-smartphone}{総務省|電気通信サービスFAQ(よくある質問)|スマートフォンとはなんですか?,\url{http://www.soumu.go.jp/main_sosiki/joho_tsusin/d_faq/faq01.html},2017年1月27日確認.}
\bibitem{1-spread}{総務省,``平成28年版情報通信白書'',2016年.}
\bibitem{1-ccc}{CCC|Fingerprint Biometrics hacked again,\url{https://www.ccc.de/en/updates/2014/ursel},2017年1月27日確認.}
\sloppy
\bibitem{1-sophos}{Chaos Computer Club claims to have “cracked” the iPhone 5s fingerprint sensor – Naked Security,\url{https://nakedsecurity.sophos.com/2013/09/22/chaos-computer-club-claims-to-have-cracked-the-iphone-5s-fingerprint-sensor/},2017年1月27日確認.}
\fussy
\bibitem{2-sakamoto}{坂本翔,``ユーザの直感的な入力をとらえるための3軸加速度センサによるジェスチャ認識の研究'',2009年度公立はこだて未来大学卒業論文.}
\bibitem{2-hamano}{\UTF{6FF5}野雅史,新井イスマイル,``加速度センサ・ジャイロセンサを併用したスマートフォンの利用認証手法の提案'',情報処理学会研究報告,Vol.2014-MBL-70,No.17,Vol2014-UBI-41,No.17,2014.}
\bibitem{3-adagrad}{John Duchi, Elad Hazan, Yoram Singer, ``Adaptive Subgradient Methods for Online Learning and Stochastic Optimization'', Journal of Machine Learning Research, 12, 2121-2159, 2011.}
\bibitem{3-adagrad-detail}{実装ノート・tiny-dnn/tiny-dnn Wiki,\url{https://github.com/tiny-dnn/tiny-dnn/wiki/実装ノート},2017年1月29日確認.}
\bibitem{3-adam}{Diederik P. Kingma, Jimmy Lei Ba, ``Adam: A Method for Stochastic Optimization'', The International Conference on Learning Representations, San Diego, 2015.}
\bibitem{3-adam-detail}{勾配降下法の最適化アルゴリズムを概観する|コンピュータサイエンス|POSTD,\url{http://postd.cc/optimizing-gradient-descent/},2017年1月29日確認.}
\sloppy
\bibitem{3-sensor-coordinate}{Sensor Coordinate System|Android Developers,\url{https://developer.android.com/guide/topics/sensors/sensors_overview.html#sensors-coords},2017年1月30日確認.}
\bibitem{3-sensor-delay}{Monitoring Sensor Events|Android Developers,\url{https://developer.android.com/guide/topics/sensors/sensors_overview.html#sensors-monitor},2017年1月30日確認.}
\bibitem{4-colt-project}{Colt Project,\url{https:/dst.lbl.gov/ACSSoftware/colt/},2017年1月31日確認.}
\bibitem{4-parallel-colt}{Parallel Colt,\url{https://sites.google.com/site/piotrwendykier/software/parallelcolt},2017年1月31日確認.}
\bibitem{4-jtransforms}{JTransforms,\url{https://sites.google.com/site/piotrwendykier/software/jtransforms},2017年1月31日確認.}
\bibitem{4-cuda}{CUDA Zone|NVIDIA Developer,\url{https://developer.nvidia.com/cuda-zone},2017年1月31日確認.}
\bibitem{4-jni}{Java Native Interface Specification,\url{http://docs.oracle.com/javase/7/docs/technotes/guides/jni/spec/jniTOC.html},2017年1月31日確認.}
\fussy
\bibitem{5-log}{Log|Android Developers,\url{https://developer.android.com/reference/android/util/Log.html},2017年2月3日確認.}
\end{thebibliography}


\end{document}
