\chapter*{要旨}
% 従来型のフィーチャーフォンと比較して
スマートフォンの普及により,従来型のフィーチャー・フォンと呼ばれるデバイスと比較して多種多様なサービスを利用できるようになった.
例えば,スマートフォンはパーソナルコンピュータ向けに設計されたWebサイトを閲覧できるフルブラウザが利用できる.
これを用いることで,オンラインショッピングを含むサービスををどこでも手軽に利用できるようになった.
またスマートフォンは,様々な企業や個人が開発した多種多様なアプリケーションを自由にインストールし利用できる.
例えば国内銀行各社からスマートフォン向けアプリが提供されているが,これを用いることでアプリ内で口座残高の確認や入出金明細の確認はもちろん,振り込みや振り替えなども行える.

上記のようなサービスを利用する際には,基本的にあらかじめ登録したユーザ名とパスワードを用いた個人認証を行うのが一般的だが,これら情報をアプリ内に記憶しておくことで利用毎に再入力する手間を省けるような仕組みを持つ場合もある.

% スマートフォンデバイスを利用する上でのセキュリティリスクとその対策
このようにスマートフォンによって日々の生活がより豊かになった一方で,デバイスがより多くの個人情報を内包するようになった.
第三者によって不正に個人情報にアクセスされたりインストールされたアプリを通じて様々なサービスをなりすまし利用されないように,デバイス利用時にはパスコード認証や指紋認証などを用いて本来の端末所有者であるか認証するよう設定することが推奨されている.

% 既存の認証方式の欠点
現在スマートフォンにおいて個人の認証方式として広く使われている方法として,パスコード認証と指紋認証が挙げられる.
しかしながらこれら認証方式にはいくつかの問題点が考えられる.
まずパスコード認証では認証作業が煩雑であったり認証に用いる鍵情報の自由度が低いという点がある.
また指紋認証では指紋を読み取るためのハードウェアが必要である点や,指紋情報は変更ができないため,何らかの原因でこの情報が第三者に漏洩した場合はその指紋を用いた個人認証ができなくなるという点がある.

% 本研究の目的と,その実現方法
そこで本研究ではスマートフォンに一般的に搭載されている加速度センサと角速度センサを利用し,人間の動きを用いて端末を振ることで個人を認証するシステムを開発した.
本システムでは,認証時に入力されたデータが本来の端末所有者本人によるものかを識別するために,人工ニューラルネットワークを利用した.
端末所有者はあらかじめ認証時に利用したい動きをシステムに入力し,登録処理を行う.
登録処理では,まずDenoising Autoencoderを用いて入力されたデータの特徴を学習し,その後Denoising Autoencoderの後ろに識別を行うニューロンを繋いで学習を行う.
認証時には,登録処理で得られた識別器を用いて入力されたデータが端末所有者本人のものであるかを識別する.

% 結論を簡単に
評価の結果,云々.
本システムにより,パスコード認証が抱える認証の煩雑さと鍵情報の自由度といった問題点を軽減した.
また指紋認証における鍵情報が変更できないという問題点を解消し,直感的に個人認証を行うことが可能になった.
