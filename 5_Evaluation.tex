\chapter{評価}
本章では,本研究で開発した個人認証システムの認証精度と登録及び認証処理にかかる時間の評価を行う.

\section{認証精度の評価}
認証制度の評価について,端末を持ち上げるモーションを対象とする.
本システムを用いてあらかじめ6名の被験者に端末を持ち上げるモーションを4回入力してもらい,モーションデータの収集を行った.
各被験者ごとに,最初の3回分を登録モードにおける訓練データとして用い,最後の1回分を認証モードにおける入力データとして10回認証を試行した.
また,それぞれの試行時になりすまし認証データとして筆者自身が同様のモーションを入力して得たモーションデータも入力した.

この実験により得られた出力を表\ref{auth-result}に示す.


\section{登録及び認証の処理時間の評価}
本システムについて,登録及び認証の処理時間を計測した.
Androidのログ出力用APIとして用意されているLogクラス\cite{5-log}を用い,モーション入力が終了し計算処理中であることをユーザに示すプログレスダイアログが表示される部分と,登録及び認証処理が終わり処理結果が表示される前にプログレスダイアログが非表示となる部分にログ出力を行うコードを挿入した.
このログから出力される時刻情報を用い,登録及び認証をそれぞれ10回試行した平均処理時間を求めて評価を行った.
また,認証モードについては端末ローカルでの計算処理も行えるため,こちらの平均処理時間も求めることとする.

実験結果を表\ref{time-result}に示す.
