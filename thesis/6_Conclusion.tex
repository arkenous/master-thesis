% @suppress
\chapter{結論}
本研究では,一般的なスマートフォンに搭載されている加速度センサと角速度センサを利用し,端末を振る動きで個人認証を行うシステムを開発した.
そして,複数の被験者を対象にした認証精度の評価と登録及び認証の処理時間計測を行った.
認証精度の評価実験から,多くの被験者については本人のモーションデータとなりすまし認証によるモーションデータの識別がある程度明確にできたが,一部の被験者はなりすまし認証のデータで得られた識別器の出力がより低く出ており,上手く識別できていない.
また,本人のモーションデータとなりすまし認証によるモーションデータの識別ができた被験者についても,試行によっては識別できなくなるなどの不安定さが見られた.
本システムでは,端末所有者のモーションデータとなりすまし認証によるモーションデータを識別するために,端末所有者が入力したモーションデータの30\%を0で上書きしたダミーデータを用いた.
だが,端末所有者が入力したモーションデータに0に近い値が多く含まれる場合は端末所有者が入力したモーションデータとダミーデータの間の差が出ない可能性が,モーションデータの値域が広く変動が激しい場合は0で上書きすることでダミーデータとの差が出過ぎる可能性がある.
これによって,端末所有者が入力したモーションデータであってもなりすまし認証であると識別されたり,なりすまし認証によるモーションデータであっても端末所有者の入力であると識別される場合があったのではないかと考えている.

本システムでは端末所有者が入力したモーションデータに0を上書きする際はランダムにデータを選択しているが,何らかの規則に従って0で上書きするかを決めることにより,識別器の不安定さを解消できる可能性がある.
加えて,ダミーデータの生成方法として0でデータを上書きする方法ではなく,何らかの別の方法を用いることができないかを検討する必要がある.

% また,望む機能を実現するためにどのような技術が他にあるか(畳み込みニューラルネットワークとか)
本システムではダミーデータを用いることで端末所有者のモーション入力を識別することを目指したが,畳み込みニューラルネットワークのようなより高度な人工ニューラルネットワークを用いて端末所有者の端末の振り癖を学習させることで,特定のモーションに縛られない自由度と精度の高い個人識別が実現できる可能性があるため,検討する必要がある.
